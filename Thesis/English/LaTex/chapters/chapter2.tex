\section{Preface}
There have been various projects with different goals since the first crypto currency or bitcoin was built. The purpose of using this technology is to enhance transparency, security and speed and is commonly used in services where these three factors are particularly important.
Bitcoin, for example, uses all three factors. Apps like Spotify and Steam have used its transparency feature in their apps. And programs like Storeke have borrowed security from this technology.
\section{Examples of smart contract applications in Blockchain}
This Technology is being used by so many branches and it speeds off the functionality of the organizations that its being used in in this section we analyze some of those sections.

\subsection{In medical}
The technology has also been used in various fields and disciplines; personal health records can be encrypted and stored with a private key and accessed only by specific individuals. The same strategy can be used to ensure that investigations are conducted through HIPAA rules (in a secure and confidential way). Surgical receipt can be kept in chains and sent automatically to insurers as proof of delivery. It can also be used to manage public health care such as medication monitoring, compliance, test results, and management of health care resources.
\subsection{In Music Industry}

The 2017 Hype Cycle report (Gartner Inc. 2017) suggests that the technology is five to 10 years away
to mainstream adoption. The report expects that during these time, some focus will be given to create
convergence in architectural styles (private and public) resulting in all distributed ledgers having
similar functional characteristics. Moreover, it is also states that “concerns remain about the viability
of the technologies, security (software and hardware), scalability, legality and interoperability”,
especially given that public ledgers seem inappropriate for most enterprises internal information.
However, even with the technology in its initial stages of development, for the purposes of
disintermediation in the music industry, it shows great immediate potential. Two of the main issues
identified in our research are:
1. the lack of access to transactional information; and
2. the inefficiencies associated to royalty payments;
Blockchain technology can solve both of these issues, while maintaining transparency
throughout the entire chain. Not only that, as we will discuss in section 5, the possibilities for
applications go beyond the traditional business models in the industry, enabling a closer relationship
between Artists and Consumers. 
For the music industry specifically, the addition of blockchain powered models may result in a
complete change of the industry’s structure. Some of the main issues identified by the analysis of its
supply chain models were the lack of transparency through the chain, musician’s low bargaining
power, and inefficiency of the royalty payments systems. Through the use of applications such as
record keeping, smart contracts and metadata analysis, these issues may be eliminated and
intermediation turned obsolete.
It is important to note, however, that even with all the potential benefits to the music industry, the
new business models introduced with the technology are also in their initial stages. The mere fact that
they function through the use of cryptocurrencies, not money, may still be too controversial to some consumers. Due to the novelty of blockchain and the platforms powered by the network, the ability
to project the success of these models is quite limited.
Another point worth mention is the fact that without a revenue stream, it is unclear how these music
platforms will continue to develop and reach consumers with the same power as their competitors
Spotify and Apple Music. All three analysed platforms (Musicoin, Resonate and Ujo) have mainly
independent artists and labels in their catalogues, making it unlikely a scenario in which those artists
would have the means to finance the projects. At the same time, by charging for the use of the
platforms, there would be no disintermediation, only an exchange of it. With time, it is possible that
the value captured by the artists would go back to decrease and intermediation power to increase.
Nevertheless, it is undeniable that blockchain technology can bring innovative business models to
the music industry. The expectation is that new enterprises start to develop applications using the
technology and increasing the total amount of usage for corporations and consumers. Furthermore,
the involvement of mainstream and independent artists in this process may be of great contribution
for the dissemination of blockchain in the music industry.
Due to the initial stages of development, it was difficult to access enough data regarding the
performance of the platforms already in use. Ideally, comparing the results of those blockchain
powered platforms with the initial results of their competitors using the subscription model could
indicate how far they are from mainstream consumption. Similarly, not a lot of research was applied
to blockchain in the music industry. Not only literature is limited, but also the perception of the
effectiveness of the technology to this industry in particular.
From this research, we have the intention of further evaluate the phase of development of
blockchain in the music industry by analysing it from a performance and adoption, by artists and
consumers, perspective.\textcite{sitonio2018impact}

\subsection{In governmental processes}
In the 2016 US election, Democrats and Republicans questioned the security of the voting system. The Green Party demanded recounts in Wisconsin, Pennsylvania and Michigan. Computer scientists say hackers can cheat by manipulating the electronic system as a result. The manual prevents fraud during encryption. Private individuals can confirm that their votes have been counted and confirm who they voted for. This system saves only the government.
According to a 2013 report from McKinsey and Company, open data - open source government data available to all citizens through the Internet - provides a platform for fraudsters to use its data to fraud, allowing farmers to track time. Carefully harvest the crop in the field, and parents can research the side effects of the drug for their sick children. Currently, this data is only published once a year and is largely unaccountable for citizen input. As a public notebook, the block chain system can open this data to citizens whenever and wherever it wants.

\section{Examples of block chain powered} government applications
Block Chain technology can be used for any transaction or information exchange that takes place in which the government is involved. The fundamental characteristics of this technology enables implementation in a wide range of processes for asset registry, inventory, and information exchange, both hard assets like physical property, and intangible assets like votes, patents, ideas, reputation, intention, health data, information etc. (Swan, 2015). The essence of a BC is that organizations can keep track of a ‘ledger’ and that organizations jointly create, evolve and keep track of one immutable history of transactions and determine successive events.
Governments from all over the world are conducting pilots using BCT. Government BC applications are diverse in nature and include digital identity, the storing of judicial decisions, financing of school buildings and tracing money, marital status, e-voting, business licenses, passports, criminal records and even tax records (see for examples Blockchain Projects, 2017). We recommend further research to compare the variety of initiatives and to analyze the source of benefits.
The BC technology requires situations in which multiple parties are involved in a transaction. A notable example is granting permits to the organizers of mass events, like concerts and demonstrations, which requires the municipality, police, fire brigade and health organizations to agree and to ensure they are prepared for dealing with the mass. Another example is the transfer of car ownership. To find a car owner, the car's transaction history has to be analyzed assuming that it contains an unambiguous property identifier. The owner of the car can be identified by searching a ledger as everybody has the same view on the BC. The rule states that only the owner can sell the car. When the car is sold a transaction needs to be created in which the previous owner confirms selling the car, the new owner confirms buying the car and the bank (or another party) confirms the payment for the transfer of the ownership. Another example is keeping an overview of the authorities provided in a public organization and the ability to change the authority only if there is agreement among nodes which are classified as being higher ranked in the hierarchy. As such, BC is a technology that replaces single databases by a distributed ledger of shared information, which should result in higher security and accessibility. This difference is schematically depicted in Fig. 1. Each node in the network contains a full copy of the BC, the transactions are recorded in the ledger and each node has access to the full history of transactions. Access to the ledger can be restricted, and the number of nodes as well as the type of consensus mechanism need to be determined. These choices influence the stewardship role of government, which we address in more detail in Subsection 6.5.
A final example is the use of BCT for land title projects. This BC applications is particularly useful when ownership records are not preserved in a systematic way or the operating organization is not trusted. In some countries the ownership of a land title is hard to detect. By using a BC application every transaction of land property should be registered. BCT prevents manipulation and loss of data. The transfer of land property requires that the lawful owner has to sign, for which there should be proof of ownership, no left mortgage should rest on the land property, and a payment (money transfer) from the buying to the selling party has to be made. BC can be used to protect the rights of the owner of the land, to resolve disputes, to make sure that ownership is correctly transferred and to prevent any unauthorized and fraudulent changes. However, BCT does not help to address the accuracy of the land titles, but rather seeks to clarify the authenticity of the title. In the case that input is manipulated and still complies with the conditions it will nevertheless be accepted by the network and added to the BC. Hence BC can be used as one of the instruments to fight corruption with land registries, but should be part of a wider institutional setting including other instruments for a legally correct and compliant land registry administration.
These examples show that BC applications can have significant effects on the way organizational processes are designed. E.g. in the case of using a BC applications for land registry organizations involved in land registry processes can directly interact with each other. This reduces the mediating role of the land register organizations who only need to focus on developing, maintaining and governing the BC application. Yet, if and how such organizations should be transformed to serve as owners and guardians of the BC application is still an open question. To the best of our knowledge, there are no deep analyses of these changing administrative processes and organizations in their institutional context yet and research in this area is required.
Some authors even go one step further by arguing that BC is “an institutional technology of governance that competes with other economic institutions of capitalism, namely firms, markets, networks, and even governments” (Davidson, De Filippi, and Potts, 2016, p.1). Atzori (2015) even stated that BC can be viewed as technology that competes with the role of government in society. Technology competing with an institution might be considered as a technology-push, far-fetched and naive, but nevertheless such propositions should not be ignored and research is needed to position this in a more realistic view which takes into account both technical and institutional elements. What the BCT has to offer is that instead of transactions being handled directly by government organizations, they can be handled by distributed ledger technology running on P2P platforms that are enabled and facilitated by (or on behalf of) government organizations. This raises questions about who will set-up, execute and maintain these architectures which will likely still be the role of government, but the actual transactions might be performed without the government.\textcite{olnes2017blockchain}

\subsection{Public and community value}
The collection can facilitate self-organization by providing a self-management platform for companies, NGOs, foundations, government agencies, academics and individual citizens. Parties can interact and share information globally and transparently - think of iCloud and google drive, but it's bigger and less risky.

\subsection{Assignment of responsibility}

By using blockchain, the use of responsibilities is clearly and transparently communicated to those responsible, such as students, administrators, staff, and department heads.
This will identify all work done at the academic, team and community level and encourage fine-tuned individuals and people who are doing their job properly.

\subsection{Identity in cyberspace}
Whether we like it or not, online companies know everything about us. Some of the companies we buy online sell their identity information to advertisers who send you their ads. By creating a protected data point, Blockchain allows you to be safe from these abuses and to show everyone who needs your information only that particular area.

\subsection{Passports}
While there exists somewhat imperfect systems for establishing personal identity in the real
world, in the form of identity document, driver’s licenses and even passports, there is no
equivalent system for securing either online authentication of our personal identities or the
identity of digital entities. So while governments can issue forms of physical identification, online
identities and digital entities do not recognize national boundaries and digital identity
authentication appears at first look to be an intractable problem without an overseeing global
entity.
Blockchain technology may offer a way to circumvent this problem by delivering a secure
solution without the need for a trusted, central authority. It can be used for creating an identity on
the blockchain, making it easier to manage for individuals, giving them greater control over who
has their personal information and how they access it.
By combining the decentralized blockchain principle with identity verification, a digital ID can be
created that would act as a digital watermark which can be assigned to every online transaction.
The solution can help the organizations to check the identity on every transaction in real time,
hence, eliminating rate of fraud. Consumers will be able to login and verify payments without
having to enter any of the traditional username and password information. Through blockchain
solutions, consumers can simply use an app for authentication instead of using traditional
methods, such as a username and password. The solution will store their encrypted identity,
allowing them to share their data with companies and manage it on their own terms.\textcite{jacobovitz2016blockchain}

The first digital passport was launched in 2014 on GitHub and can help owners identify themselves online and even offline abroad. how it works? You take a picture of yourself, stamping it with a public and private key, both of which are encrypted to prove it. According to the Bitcoin address, the passport is stored in a general IP in the booklet and is verified by the users of the blockchain.
In general, this application of blockchain includes the creation of death, birth, marriage and birth certificates. It also retains its functionality for all ID cards.

\subsection{Energy supply}
In some parts of the world, businesses and homes can take advantage of blockchain-based distribution networks to store energy and accurately track consumption. Blockchain can also be used to improve the pursuit of clean energy. When electricity is sent to the grid, no one can tell if it was produced by fossil fuels or wind or solar energy.
Traditionally, renewable energy is tracked by government-approved transferable licenses. These permissions have problems that "blockchain" can solve.

\subsection{Accounting}
Transaction recording through blockchain significantly eliminates human error and saves data from possible tampering. Remember that records are approved each time a new block is generated. Of course, the whole accounting process is also more effective. Instead of keeping separate records, businesses can keep only one general ledger. The integrity of a company's financial information will also be guaranteed.

\subsection{Stock market}
The concept of using blockchain technology for security and exchange of goods has been around for some time. Due to the open and secure nature of the blockchain system, the stock market is also looking at blockchain technology as a jump point. The Australian stock market has long been planning to move its operations to a blockchain-based system, designed by the startup blockchain Digital Asset Holding.

\subsection{Internet of things}
The combination of these two technologies increases their speed and ease.

\section{Conclusion}
It is important to note that working with Black China, if these standards are met, can become a powerful tool for doing and improving business, making the global economy fairer, and helping to support more open and fair societies.
Now, with the application of this technology and the tools made with its help, it is possible to write a complete program example for the vulnerable points of the society. Blockchain is an emerging technology that requires compatible applications. A set of rules must also be set for the use of this technology. This technology must be able to prove its effectiveness in order to be widely used.

