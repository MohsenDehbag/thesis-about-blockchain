\section{Conclusions and the Future of Blockchain}
Blockchain is a new technology that must overcome a number of problems before its full advantages
can be exploited, but so was the Internet before October 1990, when Sir Tim Berners-Lee introduced
three fundamental technologies that formed the foundation of the World Wide Web (WWW) and are
utilized until today. By the end of 1990, the first web page was posted on the internet, which people
could visit and view information on, mostly transmitted through modems and regular telephone
lines. However, such information consisted of text only characters, as sounds, images and videos
were outside the communication capabilities of that time. Google, Amazon, Facebook, or YouTube
were unimaginable at that time, when even sending an email, before the Mosaic web browser was
introduced, was considered a technological achievement. Nobody should be surprised, therefore, with
blockchain’s current limitations, as it is at around the same stage as the Internet was in the mid-1990s.
In this paper we discussed the core advantages of blockchain and pointed out that its full potential
would be unleashed in the not too distant future [40], with new giant firms, the equivalent of Google,
Amazon, and Facebook, emerging to exploit the advantages of blockchain technologies. In this
conclusion, we would like to reiterate the value of blockchain and its disruptive nature while also
deliberating about its future achievements. A recent Deloitte Global Blockchain Survey  concluded
that 2019 was a turning point for blockchain when a radical shift happened in the attitudes of business
leaders who recognized that blockchain is for real and that it can serve as a pragmatic solution to
business problems across industries and use cases. That is, these leaders recognized a shift from
“blockchain tourism” and exploration towards the building of practical business applications, as
blockchain has finally entered the mainstream of business applications. Blockchain guarantees trust,
assures immutability, transparency, and supports disintermediation in addition to providing extra
security for transactions executed over the Internet. These are considerable advantages that cannot
be ignored, while its disadvantage of the cost of implementation can be depreciated and reduced in
a short amount of time, as more experience with applications is gained and blockchain becomes a
core technology. Most importantly, however, as usage increases the motivation for improvements will
increase, too, as has been the case with the Internet that witnessed substantial advancements over
a short period of time. Such advancements will provide solutions to blockchain’s inability to scale,
significantly reducing usage costs.
Blockchain’s ability towards security and immutability can also be used for storing the highly
sensitive, personal data needed to determine patterns in sensitive cases such as those involving the
healthcare sector. Furthermore, blockchain can contribute to breaking the black box of AI by tracing
how algorithms work and how their input affects the output of machine learning, while AI can increase
the efficiency of blockchain far better than humans, or standard computing. Finally, Bitcoin, viewed as
blockchain’s first innovative success , can contribute to applying the technology to additional areas,
increasing the popularity of both Bitcoin and AI, as well as their various applications. Blockchain and
AI are new technologies and much will depend on future, yet unknown, technological advancements.
However, there is considerable potential that can raise their separate, as well as their combined,
usefulness to new, high levels of value and applicability. This has been the case with the Internet, as
well as, all new technologies whose future value has been underestimated greatly at the outset.
